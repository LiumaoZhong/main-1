\chapter{Analyzing and Extending Bigram Grammars}
\label{cha:SLMath}

\begin{itemize}
    \item closure under complementation
    \item think of grammars as constraints/dependencies; can we combine dependencies? yes, closure under intersection
    \item characterization in terms of substring substitution closure
    \item generalization to n-grams; many follow-up exercises
    \item cognitive and learnability implications for local dependencies
\end{itemize}

\subsection{Basic Closure Properties}

\begin{lemma}
    The class of bigram languages is closed under intersection.
\end{lemma}
%
\begin{proof}
    We prove that for any two bigram languages generated by (positive) bigram grammars $G_1$ and $G_2$, $L(G_1) \cap L(G_2) = L(G_1 \cap G_2)$.
    We first show $L(G_1) \cap L(G_2) \subseteq L(G_1 \cap G_2)$.
    Let $w$ be an arbitrary string belonging to both $L(G_1)$ and $L(G_2)$, i.e.\ $w \in L(G_1) \cap L(G_2)$.
    Then every bigram of $w$ is contained in both $G_1$ and $G_2$, so $w \in L(G_1 \cap G_2)$.
    Since $w$ is arbitrary, we have $L(G_1) \cap L(G_2) \subseteq L(G_1 \cap G_2)$.
    The same reasoning can be applied in the other direction, yielding $L(G_1 \cap G_2) \subseteq L(G_1) \cap L(G_2)$.
    These two facts jointly imply $L(G_1) \cap L(G_2) = L(G_1 \cap G_2)$.
\end{proof}
%
% hw: spell out the second subset relation

\begin{lemma}
    The class of bigram languages is not closed under union. 
\end{lemma}
%
\begin{proof}
    We give an example of two bigram languages whose union is not a bigram language.
    Let $L_1 \is \setof{\mathit{ab}}$ and $L_2 \is \setof{\mathit{b}, \mathit{bb}, \mathit{bbb}, \ldots}$.
    Assume without loss of generality (w.l.o.g.) that all bigram grammars are positive.
    Then any bigram grammar that generates $L_1$ must contain the bigrams $\LeftEdge a$, $\mathit{ab}$, and $b \RightEdge$.
    Similarly, a bigram grammar generating $L_2$ must contain the bigrams $\LeftEdge b$, $\mathit{bb}$, and $b \RightEdge$.
    Therefore a bigram grammar that generates all strings in $L_1 \cup L_2$ must contain at least these bigrams.
    But such a grammar also generates the string $\mathit{abb}$, which is not part of $L_1 \cup L_2$.
    Hence there is no bigram grammar that generates all the strings in $L_1 \cup L_2$ and nothing else, so $L_1 \cup L_2$ is not a bigram language.
\end{proof}
%
%hw: extend proof to case where both languages are infinite

\begin{lemma}
    The class of bigram languages is not closed under (relative) complement.
\end{lemma}
%
\begin{proof}
    This is an immediate consequence of the previous lemmata and what is called De Morgan's law: $L_1 \cup L_2 = \complementof{\complementof{L_1} \cap \complementof{L_2}}$.
    If the class of bigram languages were closed under both complement and intersection, it would be closed under union, too.
    Since closure under union does not hold, closure under intersection or complement must fail.
    We have already seen that closure under intersection holds, so closure under complement cannot hold, too.
\end{proof}
%
%hw: give a proof by example
Slightly surprising because we have the option to take complements of grammars, but that's not string language complements.

\begin{lemma}
    The class of bigram languages is not closed under relabelings.
\end{lemma}
%
\begin{proof}
    Consider the language $L \is \setof{\mathit{ab}, \mathit{abab}, \mathit{ababab}, \ldots}$ and the relabeling $r$ that replaces all $b$s by $a$s.
    The image of $L$ under $r$ consists of all strings that contain a (positive) even number of $a$s and nothing else: $\mathit{aa}$, $\mathit{aaaa}$, $\mathit{aaaaaa}$, and so on.
    Every such string contains the bigrams $\LeftEdge a$, $\mathit{aa}$, and $a \RightEdge$.
    But a grammar with these bigrams also generates the strings $a$, $\mathit{aaa}$, $\mathit{aaaaa}$, and so on, which shows that the image of $L$ under $r$ cannot be generated by a bigram grammar.
    Consequently, closure under relabelings does not hold for the class of bigram languages.
\end{proof}
%
ties into abstractness debate (e.g. if we have hidden C-V layer, language is definable).

\subsection{Substring Substitution Closure}

\subsection{\texorpdfstring{Generalization to $n$-Gram Grammars}{Generalization to n-Gram Grammars}}

\section{Implications for Phonology}

%hw: implement a negative bigram grammar
%hw: implement translation from positive into negative bigram grammar

local substring substitution

%hw: what happens if we run our scanner with grammar for {LR} and string LLLRRR; is this accepted? why? how to fix the issue?
%hw: scanner is not psycholinguistically plausible (incrementality)
%hw: is memory usage a good argument for online scanner?
