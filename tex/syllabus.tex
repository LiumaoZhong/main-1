\setcounter{chapter}{-1}
\chapter{Syllabus}
\label{cha:syllabus}
\setcounter{page}{1}

\fcolorbox{gray!25}{gray!25}{%
    \centering
    \begin{tabular}{ll}
        \textbf{Course:} Computational Linguistics 2&
        \textbf{Name:} Thomas Graf\\
        \textbf{Course\#:} LIN 637 &
        \textbf{Email:} \href{mailto:lin637@thomasgraf.net}{lin637@thomasgraf.net}\\
        \textbf{Time:} MW 2:30--3:50? &
        \textbf{Office hours:} tba \\
        \textbf{Location:} CompLab SBS N250 &
        \textbf{Office:} SBS N249\\
        \textbf{Course Website:} \href{http://lin637.thomasgraf.net} {lin637.thomasgraf.net} &
        \textbf{Personal Website:} \href{http://thomasgraf.net}{thomasgraf.net}
    \end{tabular}
}

\section{Course Outline}

\subsection{Bulletin Description}
An introduction to the theoretical foundation of computational linguistics.
The course emphasizes the importance of algorithms, algebra, logic, and formal language theory in the development of new tools and software applications.
Empirical phenomena in phonology and syntax are sampled from a variety of languages to motivate and illustrate the use of concepts such as strictly local string languages, tree transducers, and semirings.
Students will develop familiarity with the literature and tools of the field.

\subsection{Full Description}

This course serves a specific purpose in our program (see Fig.~\vref{fig:Syllabus_Program}):
it acts as the bridge from introductory courses in linguistics (Syntax 1, Phonology 1, Phonetics) and computational methods (Statistics, Mathematical Methods in Linguistics, Computational Linguistics 1) to advanced courses and seminars in computational\slash mathematical linguistics.
In contrast to the NLP courses offered by the department of computer science, this course focuses on studying the properties of natural language from a computationally informed perspective.
The question is not how computers can solve language-related tasks, but how language can be conceptualized as a computational problem.
This emphasis is also reflected in the selection of topics for this course.

\begin{itemize}
    \item \textbf{What this course is not about}
        \begin{itemize}
            \item Computer-assisted research methods in linguistics
            \item Software development for natural language tasks
        \end{itemize}
    \item \textbf{What is not covered but benefits from what is covered}
        \begin{itemize}
            \item Speech recognition
            \item OCR
            \item Text generation
            \item Parsing
            \item Semantic analysis
            \item Machine translation
        \end{itemize}
    \item \textbf{List of topics}
        \begin{itemize}
            \item \emph{Phonology and Morphology}
                \begin{itemize}
                    \item The role of formalization
                    \item String languages
                    \item Subregular hierarchy
                    \item Regular languages
                    \item Generative capacity of phonology
                    \item String transductions
                    \item 2-level morphology
                    \item Equivalence of SPE and OT
                \end{itemize}
            \item \emph{Syntax}
                \begin{itemize}
                    \item Tree languages
                    \item Syntax is more complex than phonology
                    \item Mildly context-sensitive formalisms (TAG, MGs)
                    \item Tree transductions
                    \item Regular representations of MCS formalisms
                    \item Reinterpreting the T-model
                \end{itemize}
        \end{itemize}
\end{itemize}

A rough outline of the course progression is given in Tab.~\vref{tab:Syllabus_CourseOutline}.
Many of the topics we cover draw from very specialized areas of formal language theory that even most mathematicians and computer scientists do not know about, e.g.\ the correspondence between finite-state machines and monadic second-order logic, or the logical characterization of tree transductions.
So this course might be of value to you even if you do not particularly care about natural language.
Make no mistake, though, we'll talk a lot about language and linguistics --- this is not a math class.

\begin{table}
    \centering
    \begin{tabular}{rp{5.5cm}p{5.5cm}}
        \toprule
        \emph{Wk} & \emph{Formal} & \emph{Linguistics} \\
        \toprule
        1         & What is computation? & Marr's Three Levels\\
        2         & Formalizing phonology & Why formalize?\\
        3         & Strictly local languages & Local dependencies\\
        4         & Subregular hierarchy & How powerful is phonology?\\
        5         & Regular languages & Abstractness\\
        6         & String transductions & SPE-OT equivalence\\
        7         & Two-level morphology & Null morphemes\\
        \midrule
        8         & (Spring Break) & \\
        \midrule
        9         &  Weak Generative Capacity & $\text{Phonology} < \text{Syntax}$\\
        10        &  Tree languages & Headedness, feature percolation\\
        11        &  Local tree languages & GPSG\\
        12        &  Recognizable tree languages & GB\\
        13        &  TAG and MGs & Minimalist syntax\\
        14        &  Tree transductions & Reinterpreting the T-model\\
        15        &  Unification via first-order logic & Strictly Derivational Minimalism\\
        \bottomrule
    \end{tabular}
\caption{Tentative course outline}
\label{tab:Syllabus_CourseOutline}
\end{table}

\begin{figure}
    \rotatebox{0}{
        \footnotesize
        \begin{tikzpicture}[
    every node/.style = { draw, thick },
    every path/.style = { ->, thick },
    sug/.style = { dashed },
    req/.style = { },
    ]
    \node[fill=gray!25] (CL2) at (0,0) [align=center] {Computational Linguistics 2\\ (Lin 637)};

    % Prereqs
    \node (Phon) [above=of CL2, xshift=-8em, align=center] {Phonology 1 (Lin 522)\\
                                                                \emph{or}\\
                                                            Phonetics (Lin 523)
                                                        };
    \node (Syntax) [left=of Phon, align=center] {Syntax 1\\ (Lin 521)};
    \node (Math)   [above=of CL2, xshift=8em, align=center] {Statistics (Lin 538)\\
                                                                \emph{or}\\
                                                            Mathematical Methods (Lin 539)
                                                        };
    \node (CL1) [right=of Math, align=center] {CompLing 1\\ (Lin 537)};

    % CS branch
    \node (NLP) [below right=of CL2, xshift= 8em, align=center] {Introduction to NLP\\ (CSE 628)};
    \node (Machine) [below=of NLP, xshift=-8em, align=center]  {Machine Learning\\ (CSE 512)};
    \node (Speech)  [below=of NLP, xshift= 8em, align=center] {Speech Processing\\ (CSE 542)};
    \node (AI) [below=of Machine, align=center] {Artificial Intelligence\\ (CSE 537)};

    % Linguistics branch
    \node (CompSem) [below=of CL2, xshift=-16em, align=center] {Computational Semantics\\ (Lin 626)};
    \node (CompPhon) [below=of CompSem, align=center] {Computational Phonology\\ (Lin 627)};
    \node (CompSyn) [below=of CompPhon, align=center] {Computational Syntax\\ (Lin 628)};

    \node (Learn) [right=of CompSem, xshift=4em, align=center]  {Learnability\\ (Lin 629)};
    \node (Parse) [right=of CompPhon, xshift=4em, align=center] {Parsing and Processing\\ (Lin 630)};

    % Branches
    \draw[sug] (Syntax) |- (CL2);
    \draw[sug] (Phon) to (CL2);
    \draw[sug] (Math) to (CL2);
    \draw[req] (CL1) |- (CL2);
    \draw[req] (CL1.south -| NLP.north) -- (NLP);
    \draw[sug] (NLP) -| (Machine);
    \draw[sug] (NLP) -| (Speech);
    \draw[sug] (NLP) |- (AI);
    \draw[sug] (Learn |- CL2.south) -- (Learn); 
    \draw[sug, transform canvas={xshift=2em}] (Parse |- CL2.south) -- (Parse);
    \draw[sug] ($(CL2.south)-(6em,0)$) |- (CompSem);
    \draw[sug] ($(CL2.south)-(5em,0)$) |- (CompPhon);
    \draw[sug] ($(CL2.south)-(4em,0)$) |- (CompSyn);
\end{tikzpicture}

    }
\caption{Computational Linguistics 2 in the curriculum (dashed lines indicate recommendations rather than prerequisites)}
\label{fig:Syllabus_Program}
\end{figure}    

\subsection{Prerequisites}

The only official prerequisite is Computational Linguistics 1 (Lin 537) or comparable programming skills in Python.
Python will be used to illustrate formal concepts, and some of the homeworks will require you to implement an algorithm or procedure in Python.
Prior experience with git and markdown is useful for the homeworks but not required.

It is also helpful to have some basic familiarity with linguistics (phonemes, phrase structure rules, syntactic trees) and mathematics (sets, functions, relations, and first-order logic as covered in Semantics 1, for instance).
You can take an online survey to identify weaknesses, and several introductory readings on these topics are available on the course website.

\medskip
\noindent
\hspace{-.75em}
\begin{tabular}{ll}
    \textbf{Survey URL:} & 
    \href{https://testmoz.com/432409}{https://testmoz.com/432409}
\end{tabular}

\section{Teaching Goals}
\begin{itemize}
    \item \textbf{Practical Skills}
        \begin{itemize}
            \item conceptualize a problem in mathematical terms
            \item optimize your programs through the use of adequate algorithms and data structures (dynamic programming techniques, hash tables, etc.)
            \item a more abstract and theoretically informed perspective on current tools and techniques in NLP
            \item an understanding for how linguistic insights can be invoked to simplify NLP tasks
        \end{itemize}
    \item \textbf{Research Skills}
        \begin{itemize}
            \item assess linguistic phenomena from a computational perspective
            \item evaluate linguists' claims about computational efficiency
            \item basic overview of current research in theoretical computational linguistics
            \item use computational concepts to identify new empirical generalizations
            \item bring linguistic data to bear on computational claims
            \item mathematically informed understanding of linguistic theories
        \end{itemize}
\end{itemize}


\section{Grading}
\begin{itemize}
    \item \textbf{Homework}
        \begin{itemize}
            \item exercises, programming assignments, or critical evaluations of assigned readings
            \item Homework submission and grading is done via github.
            \item No late hand-ins!
            \item Collaboration on homework problems is encouraged as long as you write up the solutions by yourself, using your own words, examples, notation, and code.
        \end{itemize}
        %
    \item \textbf{Readings}
        \begin{itemize}
            \item at most two readings per week
            \item It is presupposed in the lectures that you have done the required readings.
            \item Reading comprehension may be tested as parts of the homeworks.
        \end{itemize}
        %
    \item \textbf{Lecture Note Feedback}
        \begin{itemize}
            \item I plan to publish the lecture notes as an open-access textbook with \emph{Language Science Press}.
            \item This is the last time I teach the course before the submission deadline, so I want feedback.
            \item Every week, you should file issues on Github for the relevant units, where you spot typos, suggest exercises, pictures, examples, etc.
            \item This is a collaborative enterprise: comment on other student's suggestions if you (dis)approve, expand their ideas, and so on.
        \end{itemize}
        %
    \item \textbf{Workload per Credits}
        \begin{itemize}
            \item \emph{0 credits}: none, but I highly recommend that you at least read the assigned papers as they will be important for following the lectures
            \item \emph{1 credit}: readings
            \item \emph{2 credits}: readings, feedback
            \item \emph{3 credits}: homework, readings, feedback
        \end{itemize}
\end{itemize}


\section{Online Component}

This class uses some online tools to facilitate homework collaboration and submission, student discussions, and dynamic lecture evaluation.

\begin{itemize}
    \item \textbf{Homework submission}\\
        \emph{How it works:}
        Homeworks will distributed via a github repository. 
        You can fork this repo and upload your own code, or checkout other students' forks to see how they dealt with the problem.
        In order to submit a homework you upload your solution to your fork and issue a pull request.
        After the due date, I'll upload my solution to the repository.

        \emph{Why we do it:}
        This setup mimics the modern workflow in collaborative development projects.
        Git is one of the best-known version control systems, and github is the biggest online service for hosting git repositories.
        Familiarity with version control systems is an essential job requirement for computational linguists, and it is also very helpful for academic work.
        See this discussion on Stackflow for some ideas how git can be used in conjunction with Latex:
        \href{http://stackoverflow.com/questions/6188780/git-latex-workflow}{http://stackoverflow.com/questions/6188780/git-latex-workflow}

        \emph{What you'll need:}
        A github account (the free tier is enough) and a way of uploading your code to a github repository.
        Linux users can install git via the command line, whereas Windows and Mac users should download and install the github app, which comes with a nice GUI.

    \item \textbf{Homework Feedback and Discussion}\\
        \emph{How it works:}
        Every github repository comes with an issue (= ticket) tracker.
        If you have a question or wish to discuss a topic, you can open a ticket on the main repository (note: tickets support markdown).
        I may also use tickets to leave comments on your homeworks.

        \emph{Why we do it:}
        Once again this is an essential part of modern software development.
        And it is also a lot more convenient than anything Blackboard has to offer.

        \emph{What you'll need:}
        Not much beyond the ability to navigate the github repos.

    \item \textbf{Class announcements}\\
        \emph{How it works:}
        Normal announcements (readings, due dates) are put on the course website, time-critical ones (i.e.\ class cancellations) are emailed out via Blackboard.
        
        \emph{What you'll need:}
        If you're not officially enrolled in the course for at least 0 credits, send me a message so I can add you to Blackboard.

    \item \textbf{Lecture Notes}\\
        \emph{How it works:}
        The lecture notes are made available online Monday and Wednesday before 3pm.
        You can look at them on your laptop\slash tablet or make a hardcopy before class.
        I will not bring handouts to class unless I could not upload them on time the day before.

        \emph{Why we do it:}
        Mostly because I just don't like paper.
        Also keep in mind that you can fork the lecture notes repository and include your notes directly in the Latex source files.
        Then you can compile a version of the lecture notes with your own notes already included.
\end{itemize}


\section{Policies}

\subsection{Contacting me}
\begin{itemize}
    \item Emails should be sent to \href{mailto://lin637@thomasgraf.net}{lin637@thomasgraf.net} to make sure they go to my high priority inbox.
        Disregarding this policy means late replies and is a sure-fire way to get on my bad side.
    \item Reply time < 24h in simple cases, possibly more if meddling with bureaucracy is involved.
    \item If you want to come to my office hours and anticipate a longer meeting, please email me so that we can set apart enough time and avoid collisions with other students.
\end{itemize}

\subsection{Disability Support Services}

If you have a physical, psychological, medical or learning disability that may impact your course work, please contact Student Accessibility Support Center, ECC (Educational Communications Center) Building, Room 128, (631)632-6748.
They will determine with you what accommodations, if any, are necessary and appropriate.
All information and documentation is confidential.

Students who require assistance during emergency evacuation are encouraged to discuss their needs with their professors and Student Accessibility Support Center.
For procedures and information go to the following website:
\url{http://www.stonybrook.edu/ehs/fire/disabilities}

\subsection{Academic Integrity}

Each student must pursue his or her academic goals honestly and be personally accountable for all submitted work.
Representing another person's work as your own is always wrong. Faculty is required to report any suspected instances of academic dishonesty to the Academic Judiciary.
Faculty in the Health Sciences Center (School of Health Technology \& Management, Nursing, Social Welfare, Dental Medicine) and School of Medicine are required to follow their school-specific procedures.
For more comprehensive information on academic integrity, including categories of academic dishonesty please refer to the academic judiciary website at
\url{http://www.stonybrook.edu/commcms/academic_integrity/index.html}.

\subsection{Critical Incident Management}

Stony Brook University expects students to respect the rights, privileges, and property of other people.
Faculty are required to report to the Office of University Community Standards any disruptive behavior that interrupts their ability to teach, compromises the safety of the learning environment, or inhibits students' ability to learn.
Faculty in the HSC Schools and the School of Medicine are required to follow their school-specific procedures.
Further information about most academic matters can be found in the Undergraduate Bulletin, the Undergraduate Class Schedule, and the Faculty-Employee Handbook. 

